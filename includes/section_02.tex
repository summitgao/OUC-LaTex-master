\chapter{公式、图与表}

\section{公式插入}
本节为公式 \ref{eq:example} 插入测试。

\begin{equation}
\begin{split}
I(X;Y)&=\mathcal{D}_{KL}(\mathbb{J}(X,Y)||\mathbb{P}(X)\otimes\mathbb{P}(Y)).
\end{split}
\label{eq:example}
\end{equation}

\section{图片插入}
有的同学可能听说“\LaTeX{} 只能使用 eps 格式的图片”,甚至把 jpg 格式转为 eps。
事实上,这种做法已经过时。
而且每次编译时都要要调用外部工具解析 eps,导致降低编译速度。
所以我们推荐矢量图直接使用 pdf 格式,位图使用 jpeg 或 png 格式。
\begin{figure*}[ht]
    \centering
	\includegraphics[width=0.5\textwidth]{ouclogo}
	\caption{中国海洋大学图片。}
	\label{fig:ouc}
\end{figure*}

\section{表格插入}

本校没有过多的表格格式要求,只要求表头在上即可。

\begin{table}[htb]
  \centering\small
  \caption{表号和表题在表的正上方}
  \label{tab:exampletable}
  \begin{tabular}{cl}
    \toprule
    类型   & 描述                                       \\
    \midrule
    挂线表 & 挂线表也称系统表、组织表,用于表现系统结构 \\
    无线表 & 无线表一般用于设备配置单、技术参数列表等   \\
    无线表 & 无线表一般用于设备配置单、技术参数列表等   \\
    \bottomrule
  \end{tabular}
\end{table}

\section{算法环境}
这是算法演示。关于该宏包的具体用法,请阅读宏包的官方文档。这是算法演示。关于该宏包的具体用法,请阅读宏包的官方文档。这是算法演示。关于该宏包的具体用法,请阅读宏包的官方文档。


\begin{algorithm}[htb]
\small
    \caption{发表论文}
    \begin{algorithmic}[1] %每行显示行号  
        \Require 脑子  
        \While{Ture}
            做实验
            \If {巧了}
                \State break
            \EndIf
        \EndWhile 
        \end{algorithmic}  
    \end{algorithm} 

