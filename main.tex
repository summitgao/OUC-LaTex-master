\documentclass[pdf,oneside]{oucthesis}
% pdf: 便于阅读的电子版,去除掉多余的空白页,加入超链接等。
% print:打印版本,添加空白页,便于双面打印,禁用超链接。
% count: 字数统计,使用此功能,请在编译时,添加 --shell-escape 选项,并且主文件必须为 main.tex


% ========================================
%|             论文信息填写               |
% ========================================

\title{基于唱跳说唱篮球的舞蹈练习}
\entitle{How to Master Singing, Rap, Dancing, and Basketball}
\author{蔡徐坤}
\advisor{蔡徐坤的导师}
\keywords{流行,舞蹈}
\enkeywords{Pop, Dancing}

% ========================================
%                包引用                 
% ========================================
\usepackage{booktabs}
\usepackage{amsthm}
\usepackage{algorithm}
\usepackage{algorithmicx}
\usepackage{algpseudocode}
\usepackage{caption}
\captionsetup{font={small}}

% ========================================
%                内容区域                 
% ========================================
\begin{document}

% ========================================
%                  中文摘要               
% ========================================
\begin{abstract}

出道之后,蔡徐坤大部分精力都投身于新歌的创作和专辑的打造。彼时,他需要随着NINE PERCENT在三个月内完成17场大型巡回见面会,因此写歌的时间必须“挤出来”用。洗澡时、做造型时、飞机上、两个行程间或吃饭的空隙,只要有手机、旋律,任何地方都是他的创作场所;偶尔待在录音室里,甚至成为他的喘息时间。去年,新京报记者见到他时正值午饭,化妆室里传来哼鸣声,“采访完的休息时间,我都可以写一段词。我还年轻,我觉得这都OK。”他曾表示。

而《1》的发表同样“违背”偶像市场的规律。蔡徐坤本可以每月发一首,制造更多话题。但他认为,一首首发表并不足以让外界更全面地了解他的音乐风格,“当别人都走得很快,我反而要踏踏实实一步步走。”偶尔听到舆论质疑他没有作品,蔡徐坤也曾犹豫,要不要先发一部分出来?但内心却总有个声音说,“你可以再多做几首不同风格的作品,让大家看到最全面、最好的你,而不是急于求成地去展现自己。”

在纷繁的流言蜚语中,蔡徐坤始终坚持自己把控节奏,不被潮流所左右。与蔡徐坤共事过的工作人员佩服他的“自控力”,坦言为了让外界更易接受他的音乐,团队对《1》的新歌也提过不少建议,“但他一直知道自己想要什么,很多事情也都能自己作出正确的决定。”

而在蔡徐坤出道一周年之际,他再次消失于公众视野,低调前往美国,筹备自己的首次海外公演。此次公演,已计划半年之久,但直到准备充分他才肯提上日程。公演是小型Live形式,一半DJ,一半表演。他还带来了自己的新歌,并把过去的作品做了更具现场性的改编。在美国,蔡徐坤没有给自己安排任何休闲时间,每天往返于录音棚和住所,“我最近在编曲上下了很多工夫,也重新调整了自己的录音方式,甚至在即兴创作上有了新的呈现。”

海外公演后,国内巡演也进入紧张的筹备。而音乐之外的工作,他坦言,于当下仍不在考虑范围。在蔡徐坤的节奏里,音乐才是他选择“在场”的最重要方式。

《偶像练习生》的首次登台,作为个人练习生,蔡徐坤是唯一一个大胆选择自己原创歌曲上台的人。一首《I Wanna Get Love》,蔡徐坤并不吝惜在舞台上展现性感、自如、洒脱的表演方式,“只有在舞台上,我才是真正的自己。”录制前,蔡徐坤耗费半个月精心编排新的舞蹈,在录音室反复练唱,连出场造型都精心设计了多种方案。那场表演,他成为全场第一个拿A等级的选手。然而节目播出后,外界焦点却集中于他的装扮。“重新再表演一次,我还是会这样选择。”在他看来,“性感”符合这首歌的表达,也是属于蔡徐坤的风格,舞台之外的事,他都不在意。

从小,蔡徐坤就表现出音乐天赋。家中有不少人从事与艺术相关的工作。在他一岁左右,会说整句话的时候,就开始唱歌了,见到麦克风就会跑过去抓起来哼唱,一听到音乐会情不自禁地跟着节奏摇摆。

\end{abstract}

% ========================================
%|               英文摘要                 |
% ========================================
\begin{enabstract}
After his debut, Cai devoted most of his energy to the creation of new songs and the creation of albums. At that time, he needed to complete 17 large-scale tour meetings with nine percent in three months, so the time for writing songs had to be "squeezed out". When bathing, modeling, on the plane, between two itineraries or between meals, as long as there is a mobile phone and melody, anywhere is his creation place; occasionally stay in the studio, even become his breathing time. Last year, when the reporter of the Beijing News saw him, it was lunch time, and there was a hum in the dressing room. "I can write a paragraph during the rest time after the interview. I'm still young. I think it's OK. " He once said.



The publication of "1" also "violates" the law of idol market. Cai Xukun could have written one song a month to create more topics. However, he believes that the publication of a song is not enough to give the outside world a more comprehensive understanding of his music style Occasionally heard the public doubt that he has no works, Cai Xukun also hesitated, do you want to send some first? But there is always a voice in my heart that says, "you can do more works of different styles, so that everyone can see the most comprehensive and best of you, instead of rushing to show yourself."



In the numerous rumors, Cai Xukun always insists on keeping pace and not being influenced by the trend. The staff who worked with CAI Xukun admired his "self-control" and admitted that in order to make his music more acceptable to the outside world, the team also made a lot of suggestions for the new song of 1, "but he always knows what he wants and can make the right decisions for many things."



On the first anniversary of CAI's debut, he once again disappeared from public view and went to the United States in a low-key way to prepare his first overseas performance. The performance has been planned for half a year, but he was not willing to put it on the agenda until he was fully prepared. The public performance is a small live form, half DJ, half performance. He also brought his own new songs and made more live adaptations of his past works. In the United States, Cai Xukun didn't arrange any leisure time for himself. He commuted to the studio and his residence every day. "I've worked a lot in the arrangement of music recently, and I've readjusted my recording method, and even made a new presentation in improvisation."



After the overseas public performance, the domestic tour also entered the tense preparation. Work outside of music, he admits, is still out of consideration at the moment. In CAI Xukun's rhythm, music is the most important way for him to choose "presence".



As an individual trainee, Cai Xukun is the only one who boldly chooses his own original songs to appear on the stage. "I wanna get love", Cai Xukun is not reluctant to show sexy, free and easy performance on the stage, "only on the stage, I am the real myself." Before recording, Cai Xukun spent half a month elaborately choreographing new dances, repeatedly practicing singing in the studio, and elaborately designed a variety of schemes even for appearance modeling. In that performance, he became the first player to take A-level. However, after the program was broadcast, the focus of the outside world was on his costume. "Do it again, I'll still do it." In his opinion, "sexy" is in line with the expression of this song, and also belongs to CAI Xukun's style. He doesn't care about things outside the stage.



Since childhood, Cai Xukun has shown his musical talent. Many people in the family are engaged in art related work. When he was about one year old and could speak the whole sentence, he began to sing. When he saw the microphone, he would run to it and hum. When he heard the concert, he could not help but swing with the rhythm.

\end{enabstract}

% ========================================
%|                 目录                  |
% ========================================
\tableofcontents

% ========================================
%|                 正文                  |
%|      注意!!\mainmatter 设置正文格式    |
% ========================================
\mainmatter

\chapter{简介}
\section{一级标题}
\subsection{二级标题}

2017年2月,即便前途未卜,蔡徐坤为了掌握主动权仍决定奋力一搏,向前公司提出解约,并开始了长达一年,无经济来源的“北漂”生活。“为了舞台,我愿意付出所有一切。”那段时间,蔡徐坤经常宅在家或录音棚里反复地听音乐,学音乐制作,尝试制作片段式的旋律。没有人知道蔡徐坤何时能回归舞台,很多人劝他不应该在别人都抢着拍戏时,却去做音乐,但他却为自己写下“静守己心”四个字,“可能是我比较固执。”他笑笑,“沉寂的时候每个人都会有所迷失,会不知道方向。但这也是最关键的时刻。我就是很单纯地喜欢音乐,也是这样的一份热爱帮助了我。”

\chapter{公式、图与表}

\section{公式插入}
本节为公式 \ref{eq:example} 插入测试。

\begin{equation}
\begin{split}
I(X;Y)&=\mathcal{D}_{KL}(\mathbb{J}(X,Y)||\mathbb{P}(X)\otimes\mathbb{P}(Y)).
\end{split}
\label{eq:example}
\end{equation}

\section{图片插入}
有的同学可能听说“\LaTeX{} 只能使用 eps 格式的图片”,甚至把 jpg 格式转为 eps。
事实上,这种做法已经过时。
而且每次编译时都要要调用外部工具解析 eps,导致降低编译速度。
所以我们推荐矢量图直接使用 pdf 格式,位图使用 jpeg 或 png 格式。
\begin{figure*}[ht]
    \centering
	\includegraphics[width=0.5\textwidth]{ouclogo}
	\caption{中国海洋大学图片。}
	\label{fig:ouc}
\end{figure*}

\section{表格插入}

本校没有过多的表格格式要求,只要求表头在上即可。

\begin{table}[htb]
  \centering\small
  \caption{表号和表题在表的正上方}
  \label{tab:exampletable}
  \begin{tabular}{cl}
    \toprule
    类型   & 描述                                       \\
    \midrule
    挂线表 & 挂线表也称系统表、组织表,用于表现系统结构 \\
    无线表 & 无线表一般用于设备配置单、技术参数列表等   \\
    无线表 & 无线表一般用于设备配置单、技术参数列表等   \\
    \bottomrule
  \end{tabular}
\end{table}

\section{算法环境}
这是算法演示。关于该宏包的具体用法,请阅读宏包的官方文档。这是算法演示。关于该宏包的具体用法,请阅读宏包的官方文档。这是算法演示。关于该宏包的具体用法,请阅读宏包的官方文档。


\begin{algorithm}[htb]
\small
    \caption{发表论文}
    \begin{algorithmic}[1] %每行显示行号  
        \Require 脑子  
        \While{Ture}
            做实验
            \If {巧了}
                \State break
            \EndIf
        \EndWhile 
        \end{algorithmic}  
    \end{algorithm} 


\chapter{引用文献}
\section{文献引用测试章节}
学校前身是创办于1924年的私立青岛大学\cite{lamport94}。1930年,在省立山东大学和私立青岛大学的基础上成立国立青岛大学。后历经国立山东大学、山东大学时期\cite{dwx},1958年山东大学主体迁往济南,以留青的海洋系、水产系、地质系、生物系等为基础,于1959年3月成立山东海洋学院。1960年被确定为全国13所重点综合性大学之一。1988年更名为青岛海洋大学。2002年更名为中国海洋大学。

Lorem ipsum dolor sit amet\cite{xd}, consectetur adipiscing elit, sed do eiusmod tempor
incididunt ut labore et dolore magna aliqua.
Ut enim ad minim veniam, quis nostrud exercitation ullamco laboris nisi ut
aliquip ex ea commodo consequat.
Duis aute irure dolor in reprehenderit in voluptate velit esse cillum dolore eu
fugiat nulla pariatur.
Excepteur sint occaecat cupidatat non proident, sunt in culpa qui officia
deserunt mollit anim id est laborum.





% ========================================
%|                 文献                  |
% ========================================
\bibliographystyle{oucauthoryear}
\bibliography{cite}

% ========================================
%|                 致谢                  |
% ========================================
\begin{ackonwlegmentback}
在论文的最后我想向所有帮助支持过我的亲人、朋友、老师致以崇高的敬意和真诚的感谢,感谢你们在我三年研究生的生活中给予的生活和工作的支持。

2017年9月,我开始了研究生生活,时间飞逝,我即将离开学校,走向社会,在此期间,我要特别感谢XX教授,是两位老师带我进入了XXXX的世界;特别感谢实验室的同学,在我碰到问题的时候伸出援手,帮助我解决问题;最后我要特别感谢我的父母,感谢你们对我学习生涯的资助,感谢你们对我未来决定的支持。    
\end{ackonwlegmentback}

% ========================================
%|                 简历                  |
% ========================================
\begin{profile}
\section*{个人简历}
2000年0月0日出生于XX省XX市(县)。
2001年9月考入XX大学XX专业,2002年7月本科毕业并获得工学学士学位。
2002年9月考入中国海洋大学XX学院XX专业攻读硕士学位至今。

\section*{发表的学术论文}
\noindent[1] Xukun Cai, et al., Singing, Dancing, Rap, Basketball. ABC. 2019

\section*{申请发明专利}
\noindent[1]蔡徐坤。唱跳Rap篮球。PN: 20181000000.0
\end{profile}

\end{document}
